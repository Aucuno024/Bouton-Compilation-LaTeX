\documentclass[a4paper, 12pt]{article}

\usepackage{lmodern}
\usepackage[french]{babel}
\usepackage[utf8]{inputenc}
\usepackage[T1]{fontenc}
\usepackage{parskip}

\date{30 mars 2023}
\title{Sujets pour le Grand Oral}
\author{Chloé Briquet}

\begin{document}
    \maketitle
    \renewcommand{\contentsname}{Sommaire}
    \tableofcontents{}

    \section{Empaquetage convexe (mathématiques)}
        \subsection{Présentation générale}
            \indent{    }Ce sujet porte sur la question de \textbf{l'empaquetage d'un nuage de points} grâce aux mathématiques et
            l'algorithmie\footnote[1]{Sujet de MATh.en.JEANS}, il y a pour cela, dans ces domaines, plusieurs méthodes disponibles. Car bien que cela soit faisable entièrement
            à la main, il devient presque impossible de résoudre certain cas lorsque que trop de points sont présent, la représentation physique de ces points devenant alors
            \textit{presque} impossible, il sera donc obligatoire de passer par des formules mathémiques et donc pour plus d'aisance par un algorithme.

            \vspace{1cm}
        \subsection{Ouverture}
            \indent{    } Dans ce sujet nous nous intéresserons d'avantage à la partie \textsc{mathématiques} qu'algorithmique de l'empaquetage c'est à dire, comment trouver
            si un point est dans l'enveloppe convexe du nuage ou non grâce à certaines propriétés qui donc peuvent être utilisées dans un algorithme.


    \newpage


    \section{Empaquetage convexe (algoritmie)}
        \subsection{Présentation générale}
            \indent{    }Ce sujet porte sur la question de \textbf{l'empaquetage d'un nuage de points} grâce à l'algorithmie et aux mathématiques\footnote[2]{Autre point de vue du
            sujet de MATh.en.JEANS}, il y a pour cela, dans ces domaines, plusieurs méthodes disponibles. Car bien que cela soit faisable entièrement
            à la main, il devient presque impossible de résoudre certain cas lorsque que trop de points sont présent, la représentation physique de ces points devenant alors
            \textit{presque} impossible, il sera donc obligatoire de passer par des formules mathémiques et donc pour plus d'aisance par un algorithme.

            \vspace{1cm}
        \subsection{Ouverture}
            \indent{    } Dans ce sujet nous nous intéresseronsà la partie \textsc{algorithmique} plutot que mathématiques de l'empaquetage c'est à dire, comment un
            algorithme peut-il grâce à des proprités mathématiques ajouter ou non un point à l'enveloppe convexe du nuage de point.


    \newpage
\end{document}